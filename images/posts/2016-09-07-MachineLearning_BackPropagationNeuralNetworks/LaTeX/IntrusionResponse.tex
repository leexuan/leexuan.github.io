%!TEX program = pdflatex

\documentclass[a4paper, UTF8, heading = true, scheme = chinese, linespread = 1.66, titlepage]{ctexart}

%!TEX root = ./IntrusionResponse.tex

\pagestyle{empty}
\ctexset{
	punct 			    = quanjiao,
	space			      = auto,
	today			      = small,
	section 		    = {
        							beforeskip = 0pt,
        							afterskip = 0pt,
        							format += \raggedright},
	subsection 		  = {
        							beforeskip = 0pt,
        							afterskip = 0pt},
	subsubsection 	= {
        							beforeskip = 0pt,
        							afterskip = 0pt},
	tablename		    = 表,
	figurename		  = 图,
	appendixname    = 索引
}

\usepackage{comment}

\usepackage{geometry}
\geometry{top=1in, bottom=1in, left=1.25in, right=1.25in}

\usepackage{longtable,tabu}

\usepackage{amsmath}
\usepackage{amssymb}
\usepackage{bm}
\usepackage{mathrsfs}

% color define
\usepackage{color}
\usepackage{xcolor}
\usepackage{colortbl}
\def\TBLBGC{gray!30}
\def\TBLEMPHC{green!60}

\usepackage[colorlinks=true,linkcolor = {blue},citecolor = {blue!50!black}]{hyperref}
\usepackage[open]{bookmark} % located behind the usepackage of hyperref

\usepackage{cleveref}
  \crefformat{equation}  {公式~(#2#1#3)~}
  \crefformat{inequation}{不等式~(#2#1#3)~}
  \crefformat{coequation}{方程组~(#2#1#3)~}
  \crefformat{algorithm} {算法~#2#1#3~}
  \Crefformat{algorithm} {算法~#2#1#3~}
  \crefformat{figure}    {图~#2#1#3~}
  \Crefformat{figure}    {图~#2#1#3~}
  \crefformat{table}     {表~#2#1#3~}
  \Crefformat{table}     {表~#2#1#3~}
  \crefformat{Definition}{定义~#2#1#3~}
  \crefformat{Criterion} {判据~#2#1#3~}
  \crefformat{Lemma}     {引理~#2#1#3~}
  \crefformat{Example}   {例~#2#1#3~}
  \crefformat{Theorem}   {定理~#2#1#3~}
  \crefformat{Proof}     {证明~#2#1#3~}
  \crefformat{chapter}   {第#2\zhnumber{#1}#3章}
  \Crefformat{chapter}   {第#2\zhnumber{#1}#3章}
  \crefformat{section}   {第~#2#1#3~节}
  \Crefformat{section}   {第~#2#1#3~节}


% \makeatletter
% \setlength{\@fptop}{0pt}
% \setlength{\@fpsep}{0pt}
% \setlength{\@fpbot}{0pt}
% \makeatother

\usepackage{caption}
  \captionsetup[figure]{
      font = {small, bf},
      labelsep = space,
      textformat = simple,
      skip = 0pt
  }
  \captionsetup[table]{
      font = {small, bf},
      labelsep = space,
      textformat = simple,
      skip = 0pt
  }


%list environment
\usepackage{enumitem}
\setlist[1]{itemsep=-1ex, leftmargin=1.25cm}
\setlist[2]{itemsep=0ex, leftmargin=1cm, topsep = -2pt}


\usepackage{graphicx}
\usepackage{tikz, pgfplots, tkz-euclide}
\usepgfplotslibrary{external}
\tikzexternalize[prefix=Figures/ExternalizingTikZGraphics/]
%\tikzexternaldisable
\usetikzlibrary{intersections}
\usetikzlibrary{arrows.meta}


% The following codes are used to draw help lines for tikzpicture.
\makeatletter
\def\grd@save@target#1{%
  \def\grd@target{#1}}
\def\grd@save@start#1{%
  \def\grd@start{#1}}
\tikzset{
  grid with coordinates/.style={
    to path={%
      \pgfextra{%
        \edef\grd@@target{(\tikztotarget)}%
        \tikz@scan@one@point\grd@save@target\grd@@target\relax
        \edef\grd@@start{(\tikztostart)}%
        \tikz@scan@one@point\grd@save@start\grd@@start\relax
        \draw[minor help lines] (\tikztostart) grid (\tikztotarget);
        \draw[major help lines] (\tikztostart) grid (\tikztotarget);
        \grd@start
        \pgfmathsetmacro{\grd@xa}{\the\pgf@x/1cm}
        \pgfmathsetmacro{\grd@ya}{\the\pgf@y/1cm}
        \grd@target
        \pgfmathsetmacro{\grd@xb}{\the\pgf@x/1cm}
        \pgfmathsetmacro{\grd@yb}{\the\pgf@y/1cm}
        \pgfmathsetmacro{\grd@xc}{\grd@xa + \pgfkeysvalueof{/tikz/grid with coordinates/major step}}
        \pgfmathsetmacro{\grd@yc}{\grd@ya + \pgfkeysvalueof{/tikz/grid with coordinates/major step}}
        \foreach \x in {\grd@xa,\grd@xc,...,\grd@xb}
        \node[anchor=north] at (\x,\grd@ya) {\pgfmathprintnumber{\x}};
        \foreach \y in {\grd@ya,\grd@yc,...,\grd@yb}
        \node[anchor=east] at (\grd@xa,\y) {\pgfmathprintnumber{\y}};
      }
    }
  },
  minor help lines/.style={
    help lines,
    step=\pgfkeysvalueof{/tikz/grid with coordinates/minor step}
  },
  major help lines/.style={
    help lines,
    line width=\pgfkeysvalueof{/tikz/grid with coordinates/major line width},
    step=\pgfkeysvalueof{/tikz/grid with coordinates/major step}
  },
  grid with coordinates/.cd,
  minor step/.initial=.2,
  major step/.initial=1,
  major line width/.initial=2pt,
}
\makeatother

%define new environment
% \newenvironment{definition}[1][定义]{\begin{trivlist}
% \item[\hskip \labelsep {\bfseries #1}]}{\end{trivlist}}
\newtheorem{definition}{定义}


\newif\ifUseComments
\UseCommentstrue %\UseCommentstrue-打印注释,\UseCommentsfalse-不打印注释
\definecolor{ao(english)}{rgb}{0.0, 0.5, 0.0}
\newcommand{\comments}[1]{
    \ifUseComments
    {\scriptsize
    \color{ao(english)}
    \begin{itemize}[label = \%, itemsep=-1ex, leftmargin=3ex,topsep = -2pt]
        #1
    \end{itemize}}
    \fi
}

\newcommand{\demonstrate}[1]{\textcolor{red}{#1}}
\newcommand{\futureworks}[1]{
    {\scriptsize\color{red}
        #1
    }
}



\begin{document}
如图所示三层神经网络:输入层~$Layer_1$、隐含层~$Layer_2$、输出层~$Layer_3$.输入层包括三个神经元,即输入为~$\bm{x}=\{x_1,x_2,x_3\}$,样本数据集为~~$\mathcal{D}=\{(\bm{x}_i,y_i)\}$,目标输出为~$o^{\text{T}}_1$~和~$o^{\text{T}}_2$.$\omega^{\text{L}_k}_{ij}$~为第~$k-1$~层的第~$i$~个神经元到第~$k$~层的第~$j$~个神经元之间的权重.$\bm{b}^{\text{L}_k}=\{b^{\text{L}_k}_{i}\}$~为第~$k$~层神经元偏移量集合,$b^{\text{L}_k}_{j}$~为第~$k$~层第~$i$~个神经元的偏移量.$net^{\text{L}_k}_{i}$~表示第~$k$~层网络的第~$i$~个神经元的输入,$o^{\text{L}_k}_{i}$~表示第~$k$~层网络的第~$i$~个神经元的输出,$f^{\text{L}_k}_{i}(\cdot)$~表示第~$k$~层网络的第~$i$~个神经元的激活函数.



假设激活函数为~sigmoid~函数:
\begin{equation}
f(x) = \frac{1}{1+e^{-x}}
\end{equation}
其导数为:
\begin{equation}
f'(x) = f(x)(1-f(x))
\end{equation}


对于~$Layer_2$~层第~$j$~个神经元的输出:
\begin{equation}
net^{\text{L}_2}_{j} = \sum_{i=1}^{3}{\omega^{\text{L}_1}_{ij} \times x_i} + b^{\text{L}_2}_{j}
\end{equation}

对于~$Layer_3$~层的第~$j$~个神经元的输出:
\begin{equation}
net^{\text{L}_3}_{j} = \sum_{i=1}^{2}{\omega^{\text{L}_2}_{ij} \times o^{\text{L}_2}_i} + b^{\text{L}_3}_{j}
\end{equation}


第~$\text{L}_i$~层的第~$j$~个神经元的输出:
\begin{equation}
o^{\text{L}_i}_{j} = f^{\text{L}_i}_{j}{(net^{\text{L}_i}_j)} = \frac{1}{1+e^{-net^{\text{L}_i}_j}}
\end{equation}

输出的总误差为:
\begin{equation}
E_{\text{Total}} = \frac{1}{2}\sum_{j=1}^{2}{(o^{\text{T}}_j - o^{\text{L}_3}_{j})^2} = \frac{1}{2}((o^{\text{T}}_1 - o^{\text{L}_3}_{1})^2 + (o^{\text{T}}_2 - o^{\text{L}_3}_{2})^2)
\end{equation}
是~$o^{\text{L}_3}_{1}$~和~$o^{\text{L}_3}_{2}$~的函数





\section{$Layer_2$~与~$Layer_3$~间参数调整}
调整~$Layer_2$~与~$Layer_3$~间的权重~$\omega^{\text{L}_2}_{ij}$
\begin{align}
\frac{\partial E_{\text{Total}}}{\partial \omega^{\text{L}_2}_{ij}}
&=
\frac{\partial E_{\text{Total}}}{\partial o^{\text{L}_3}_{j}} \cdot \frac{\partial o^{\text{L}_3}_{j}}{\partial net^{\text{L}_3}_{j}} \cdot \frac{\partial net^{\text{L}_3}_{j}}{\partial \omega^{\text{L}_2}_{ij}} \\
&=
-(o^{\text{T}}_{j} - o^{\text{L}_3}_{j}) \cdot f^{\text{L}_3}_{j}{(net^{\text{L}_3}_j)}(1 - f^{\text{L}_3}_{j}{(net^{\text{L}_3}_j)}) \cdot o^{\text{L}_2}_{i} \\
&=
-(o^{\text{T}}_{j} - o^{\text{L}_3}_{j}) \cdot o^{\text{L}_3}_{j}(1 - o^{\text{L}_3}_{j}) \cdot o^{\text{L}_2}_{i} \\
\end{align}

令~$\delta^{\text{L}_2}_{ij}$~表示~$Layer_2$~层第~$i$~个神经元与~$Layer_3$~层第~$j$~个神经元间权重的梯度项,则
\begin{align}
\delta^{\text{L}_2}_{ij} &= \frac{\partial E_{\text{Total}}}{\partial o^{\text{L}_3}_{j}} \cdot \frac{\partial o^{\text{L}_3}_{j}}{\partial net^{\text{L}_3}_{j}} \\
&= -(o^{\text{T}}_{j} - o^{\text{L}_3}_{j}) \cdot o^{\text{L}_3}_{j}(1 - o^{\text{L}_3}_{j})
\end{align}
可以看出,~$Layer_2$~层第~$i$~个神经元与~$Layer_3$~层第~$j$~个神经元间权重的梯度项是与~$i$~无关的,故连接至~$Layer_3$~层第~$j$~个神经元相对应的权重的梯度向均为~$\delta^{\text{L}_2}_{ij}$,用~$\delta^{\text{L}_2}_{\cdot j}$~表示,
\begin{equation}
\delta^{\text{L}_2}_{\cdot j} = \delta^{\text{L}_2}_{ij} = -(o^{\text{T}}_{j} - o^{\text{L}_3}_{j}) \cdot o^{\text{L}_3}_{j}(1 - o^{\text{L}_3}_{j})
\end{equation}

则整体误差~$E_{\text{Total}}$~对~$\omega^{\text{L}_2}_{ij}$~的偏导数公式写作:
\begin{equation}
\frac{\partial E_{\text{Total}}}{\partial \omega^{\text{L}_2}_{ij}}
= \delta^{\text{L}_2}_{\cdot j} \cdot o^{\text{L}_2}_{i}
\end{equation}

则权重~$\omega^{\text{L}_2}_{ij}$~的学习公式为:
\begin{equation}
\widehat{\omega}^{\text{L}_2}_{ij} = \omega^{\text{L}_2}_{ij} - \eta \cdot \frac{\partial E_{\text{Total}}}{\partial \omega^{\text{L}_2}_{ij}} 
= \omega^{\text{L}_2}_{ij} - \eta \cdot \delta^{\text{L}_2}_{\cdot j} \cdot o^{\text{L}_2}_{i}
\end{equation}
$\eta$~为学习速率

$Layer_3$~的第~$j$~个神经元的偏移量的梯度为
\begin{align}
\frac{\partial E_{\text{Total}}}{\partial b^{\text{L}_3}_{j}}
&= \frac{\partial E_{\text{Total}}}{\partial o^{\text{L}_3}_{j}} \cdot \frac{\partial o^{\text{L}_3}_{j}}{\partial net^{\text{L}_3}_{j}} \cdot \frac{\partial net^{\text{L}_3}_{j}}{\partial b^{\text{L}_3}_{j}} \\
&= -(o^{\text{T}}_{j} - o^{\text{L}_3}_{j}) \cdot f^{\text{L}_3}_{j}{(net^{\text{L}_3}_j)}(1 - f^{\text{L}_3}_{j}{(net^{\text{L}_3}_j)}) \cdot 1 \\
&= -(o^{\text{T}}_{j} - o^{\text{L}_3}_{j}) \cdot o^{\text{L}_3}_{j}(1 - o^{\text{L}_3}_{j}) \\
&= \delta^{\text{L}_2}_{\cdot j}
\end{align}

偏移~$b^{\text{L}_3}_{j}$~的学习公式为:
\begin{equation}
\hat{b}^{\text{L}_3}_{j} = b^{\text{L}_3}_{j} - \eta \cdot \frac{\partial E_{\text{Total}}}{\partial b^{\text{L}_3}_{j}} \\
= b^{\text{L}_3}_{j} - \eta \cdot \delta^{\text{L}_2}_{\cdot j}
\end{equation}

例如,对于权重~$\omega^{\text{L}_2}_{11}$~的更新学习公式为:
\begin{align}
\widehat{\omega}^{\text{L}_2}_{11} &= \omega^{\text{L}_2}_{11} - \eta \cdot \frac{\partial E_{\text{Total}}}{\partial \omega^{\text{L}_2}_{ij}} \\
&= \omega^{\text{L}_2}_{11} - \eta \cdot \delta^{\text{L}_2}_{11} \cdot o^{\text{L}_2}_{1} \\
&= \omega^{\text{L}_2}_{11} - \eta \cdot \big(-(o^{\text{T}}_{1} - o^{\text{L}_3}_{1}) \cdot o^{\text{L}_3}_{1}(1 - o^{\text{L}_3}_{1})\big) \cdot o^{\text{L}_2}_{1} \\
&= \omega^{\text{L}_2}_{11} + \eta \cdot (o^{\text{T}}_{1} - o^{\text{L}_3}_{1}) \cdot o^{\text{L}_3}_{1}(1 - o^{\text{L}_3}_{1}) \cdot o^{\text{L}_2}_{1} 
\end{align}
例如,对于偏移~$b^{\text{L}_3}_{1}$~的更新学习公式为:
\begin{align}
\hat{b}^{\text{L}_3}_{1} &= b^{\text{L}_3}_{1} - \eta \cdot \frac{\partial E_{\text{Total}}}{\partial b^{\text{L}_3}_{1}} \\
&= b^{\text{L}_3}_{1} + \eta \cdot (o^{\text{T}}_{1} - o^{\text{L}_3}_{1}) \cdot o^{\text{L}_3}_{1}(1 - o^{\text{L}_3}_{1})
\end{align}


\section{$Layer_1$~与~$Layer_2$~间参数调整}
调整~$Layer_1$~与~$Layer_2$~间的权重~$\omega^{\text{L}_1}_{ij}$
\begin{align}
\frac{\partial E_{\text{Total}}}{\partial \omega^{\text{L}_1}_{ij}}
&= (\sum_{k=1}^{2}{\frac{\partial E_{\text{Total}}}{\partial o^{\text{L}_3}_{k}} \frac{\partial o^{\text{L}_3}_{k}}{\partial net^{\text{L}_3}_k} \frac{\partial net^{\text{L}_3}_k}{\partial o^{\text{L}_2}_{j}}})  \cdot \frac{\partial o^{\text{L}_2}_{j}}{\partial net^{\text{L}_2}_j} \cdot \frac{\partial net^{\text{L}_2}_j}{\partial \omega^{\text{L}_1}_{ij}} \\
&= (\sum_{k=1}^{2}{\delta^{\text{Layer}_2}_{\cdot k} \omega^{\text{L}_2}_{3k}}) \cdot f^{\text{L}_2}_{j}{(net^{\text{L}_2}_j)}(1 - f^{\text{L}_2}_{j}{(net^{\text{L}_2}_j)}) \cdot x_i \\
&= (\sum_{k=1}^{2}{\delta^{\text{Layer}_2}_{\cdot k} \omega^{\text{L}_2}_{3k}}) \cdot o^{\text{L}_2}_{j}(1 - o^{\text{L}_2}_{j}) \cdot x_i \\
\end{align}

令~$\delta^{\text{L}_1}_{ij}$~表示~$Layer_1$~层第~$i$~个神经元与~$Layer_2$~层第~$j$~个神经元间权重的梯度项,则
\begin{align}
\delta^{\text{L}_1}_{ij} &= (\sum_{k=1}^{2}{\frac{\partial E_{\text{Total}}}{\partial o^{\text{L}_3}_{k}} \frac{\partial o^{\text{L}_3}_{k}}{\partial net^{\text{L}_3}_k} \frac{\partial net^{\text{L}_3}_k}{\partial o^{\text{L}_2}_{j}}})  \cdot \frac{\partial o^{\text{L}_2}_{j}}{\partial net^{\text{L}_2}_j} \\
&= (\sum_{k=1}^{2}{\delta^{\text{Layer}_2}_{\cdot k} \omega^{\text{L}_2}_{3k}}) \cdot o^{\text{L}_2}_{j}(1 - o^{\text{L}_2}_{j})
\end{align}
可以看出,~$Layer_1$~层第~$i$~个神经元与~$Layer_2$~层第~$j$~个神经元间权重的梯度项是与~$i$~无关的,故连接至~$Layer_2$~层第~$j$~个神经元相对应的权重的梯度向均为~$\delta^{\text{L}_1}_{ij}$,用~$\delta^{\text{L}_1}_{\cdot j}$~表示,
\begin{equation}
\delta^{\text{L}_1}_{\cdot j} = \delta^{\text{L}_1}_{ij} = (\sum_{k=1}^{2}{\delta^{\text{Layer}_2}_{\cdot k} \omega^{\text{L}_2}_{3k}}) \cdot o^{\text{L}_2}_{j}(1 - o^{\text{L}_2}_{j})
\end{equation}

则整体误差~$E_{\text{Total}}$~对~$\omega^{\text{L}_1}_{ij}$~的偏导数公式写作:
\begin{equation}
\frac{\partial E_{\text{Total}}}{\partial \omega^{\text{L}_1}_{ij}}
= \delta^{\text{L}_1}_{\cdot j} \cdot x_i
\end{equation}

则权重~$\omega^{\text{L}_1}_{ij}$~的学习公式为:
\begin{equation}
\widehat{\omega}^{\text{L}_1}_{ij} = \omega^{\text{L}_1}_{ij} - \eta \cdot \frac{\partial E_{\text{Total}}}{\partial \omega^{\text{L}_1}_{ij}} 
= \omega^{\text{L}_1}_{ij} - \eta \cdot \delta^{\text{L}_1}_{\cdot j} \cdot x_i
\end{equation}

\begin{align}
\frac{\partial E_{\text{Total}}}{\partial b^{\text{L}_2}_{j}}
&= (\sum_{k=1}^{2}{\frac{\partial E_{\text{Total}}}{\partial o^{\text{L}_3}_{k}} \frac{\partial o^{\text{L}_3}_{k}}{\partial net^{\text{L}_3}_k} \frac{\partial net^{\text{L}_3}_k}{\partial o^{\text{L}_2}_{j}}})  \cdot \frac{\partial o^{\text{L}_2}_{j}}{\partial net^{\text{L}_2}_j} \cdot \frac{\partial net^{\text{L}_2}_j}{\partial b^{\text{L}_2}_{j}} \\
&= (\sum_{k=1}^{2}{\delta^{\text{Layer}_2}_{\cdot k} \omega^{\text{L}_2}_{3k}}) \cdot f^{\text{L}_2}_{j}{(net^{\text{L}_2}_j)}(1 - f^{\text{L}_2}_{j}{(net^{\text{L}_2}_j)}) \cdot 1 \\
&= (\sum_{k=1}^{2}{\delta^{\text{Layer}_2}_{\cdot k} \omega^{\text{L}_2}_{3k}}) \cdot o^{\text{L}_2}_{j}(1 - o^{\text{L}_2}_{j}) \\
&= \delta^{\text{L}_1}_{\cdot j}
\end{align}

偏移~$b^{\text{L}_2}_{j}$~的学习公式为:
\begin{equation}
\hat{b}^{\text{L}_2}_{j} = b^{\text{L}_2}_{j} - \eta \cdot \frac{\partial E_{\text{Total}}}{\partial b^{\text{L}_2}_{j}} \\
= b^{\text{L}_2}_{j} - \eta \cdot \delta^{\text{L}_1}_{\cdot j}
\end{equation}




例如,对于权重~$\omega^{\text{L}_1}_{23}$~的更新学习公式为:
\begin{align}
\widehat{\omega}^{\text{L}_1}_{23} &= \omega^{\text{L}_2}_{23} - \eta \cdot \frac{\partial E_{\text{Total}}}{\partial \omega^{\text{L}_1}_{23}} \\
&= \omega^{\text{L}_2}_{23} - \eta \cdot \big((\sum_{j=1}^{2}{\frac{\partial E_{\text{Total}}}{\partial o^{\text{L}_3}_{j}} \frac{\partial o^{\text{L}_3}_{j}}{\partial net^{\text{L}_3}_j} \frac{\partial net^{\text{L}_3}_j}{\partial o^{\text{L}_2}_{3}}})  \cdot \frac{\partial o^{\text{L}_2}_{3}}{\partial net^{\text{L}_2}_3} \cdot \frac{\partial net^{\text{L}_2}_3}{\partial \omega^{\text{L}_1}_{23}}\big) \\
&= \omega^{\text{L}_2}_{23} - \eta \cdot (\sum_{j=1}^{2}{\delta^{\text{Layer}_2}_{\cdot j} \omega^{\text{L}_2}_{3j}}) \cdot f^{\text{L}_2}_{3}{(net^{\text{L}_2}_3)}(1 - f^{\text{L}_2}_{3}{(net^{\text{L}_2}_3)}) \cdot x_2 \\
&= \omega^{\text{L}_2}_{23} - \eta \cdot (\sum_{j=1}^{2}{\delta^{\text{Layer}_2}_{\cdot j} \omega^{\text{L}_2}_{3j}}) \cdot o^{\text{L}_2}_{3}(1 - o^{\text{L}_2}_{3}) \cdot x_2
\end{align}
例如,对于偏移~$b^{\text{L}_2}_{3}$~的更新学习公式为:
\begin{align}
\hat{b}^{\text{L}_2}_{3} &= b^{\text{L}_3}_{1} - \eta \cdot \frac{\partial E_{\text{Total}}}{\partial b^{\text{L}_2}_{3}} \\
&= b^{\text{L}_2}_{3} - \eta \cdot \big((\sum_{j=1}^{2}{\frac{\partial E_{\text{Total}}}{\partial o^{\text{L}_3}_{j}} \frac{\partial o^{\text{L}_3}_{j}}{\partial net^{\text{L}_3}_j} \frac{\partial net^{\text{L}_3}_j}{\partial o^{\text{L}_2}_{3}}})  \cdot \frac{\partial o^{\text{L}_2}_{3}}{\partial net^{\text{L}_2}_3} \cdot \frac{\partial net^{\text{L}_2}_3}{\partial b^{\text{L}_2}_{3}}\big) \\
&=  b^{\text{L}_2}_{3} - \eta \cdot (\sum_{j=1}^{2}{\delta^{\text{Layer}_2}_{\cdot j} \omega^{\text{L}_2}_{3j}}) \cdot f^{\text{L}_2}_{3}{(net^{\text{L}_2}_3)}(1 - f^{\text{L}_2}_{3}{(net^{\text{L}_2}_3)}) \cdot 1 \\
&= \omega^{\text{L}_2}_{23} - \eta \cdot (\sum_{j=1}^{2}{\delta^{\text{Layer}_2}_{\cdot j} \omega^{\text{L}_2}_{3j}}) \cdot o^{\text{L}_2}_{3}(1 - o^{\text{L}_2}_{3})
\end{align}

% \begin{equation}
% \end{equation}



\end{document}
