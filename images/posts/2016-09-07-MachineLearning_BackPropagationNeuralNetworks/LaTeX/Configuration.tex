%!TEX root = ./IntrusionResponse.tex

\pagestyle{empty}
\ctexset{
	punct 			    = quanjiao,
	space			      = auto,
	today			      = small,
	section 		    = {
        							beforeskip = 0pt,
        							afterskip = 0pt,
        							format += \raggedright},
	subsection 		  = {
        							beforeskip = 0pt,
        							afterskip = 0pt},
	subsubsection 	= {
        							beforeskip = 0pt,
        							afterskip = 0pt},
	tablename		    = 表,
	figurename		  = 图,
	appendixname    = 索引
}

\usepackage{comment}

\usepackage{geometry}
\geometry{top=1in, bottom=1in, left=1.25in, right=1.25in}

\usepackage{longtable,tabu}

\usepackage{amsmath}
\usepackage{amssymb}
\usepackage{bm}
\usepackage{mathrsfs}

% color define
\usepackage{color}
\usepackage{xcolor}
\usepackage{colortbl}
\def\TBLBGC{gray!30}
\def\TBLEMPHC{green!60}

\usepackage[colorlinks=true,linkcolor = {blue},citecolor = {blue!50!black}]{hyperref}
\usepackage[open]{bookmark} % located behind the usepackage of hyperref

\usepackage{cleveref}
  \crefformat{equation}  {公式~(#2#1#3)~}
  \crefformat{inequation}{不等式~(#2#1#3)~}
  \crefformat{coequation}{方程组~(#2#1#3)~}
  \crefformat{algorithm} {算法~#2#1#3~}
  \Crefformat{algorithm} {算法~#2#1#3~}
  \crefformat{figure}    {图~#2#1#3~}
  \Crefformat{figure}    {图~#2#1#3~}
  \crefformat{table}     {表~#2#1#3~}
  \Crefformat{table}     {表~#2#1#3~}
  \crefformat{Definition}{定义~#2#1#3~}
  \crefformat{Criterion} {判据~#2#1#3~}
  \crefformat{Lemma}     {引理~#2#1#3~}
  \crefformat{Example}   {例~#2#1#3~}
  \crefformat{Theorem}   {定理~#2#1#3~}
  \crefformat{Proof}     {证明~#2#1#3~}
  \crefformat{chapter}   {第#2\zhnumber{#1}#3章}
  \Crefformat{chapter}   {第#2\zhnumber{#1}#3章}
  \crefformat{section}   {第~#2#1#3~节}
  \Crefformat{section}   {第~#2#1#3~节}


% \makeatletter
% \setlength{\@fptop}{0pt}
% \setlength{\@fpsep}{0pt}
% \setlength{\@fpbot}{0pt}
% \makeatother

\usepackage{caption}
  \captionsetup[figure]{
      font = {small, bf},
      labelsep = space,
      textformat = simple,
      skip = 0pt
  }
  \captionsetup[table]{
      font = {small, bf},
      labelsep = space,
      textformat = simple,
      skip = 0pt
  }


%list environment
\usepackage{enumitem}
\setlist[1]{itemsep=-1ex, leftmargin=1.25cm}
\setlist[2]{itemsep=0ex, leftmargin=1cm, topsep = -2pt}


\usepackage{graphicx}
\usepackage{tikz, pgfplots, tkz-euclide}
\usepgfplotslibrary{external}
\tikzexternalize[prefix=Figures/ExternalizingTikZGraphics/]
%\tikzexternaldisable
\usetikzlibrary{intersections}
\usetikzlibrary{arrows.meta}


% The following codes are used to draw help lines for tikzpicture.
\makeatletter
\def\grd@save@target#1{%
  \def\grd@target{#1}}
\def\grd@save@start#1{%
  \def\grd@start{#1}}
\tikzset{
  grid with coordinates/.style={
    to path={%
      \pgfextra{%
        \edef\grd@@target{(\tikztotarget)}%
        \tikz@scan@one@point\grd@save@target\grd@@target\relax
        \edef\grd@@start{(\tikztostart)}%
        \tikz@scan@one@point\grd@save@start\grd@@start\relax
        \draw[minor help lines] (\tikztostart) grid (\tikztotarget);
        \draw[major help lines] (\tikztostart) grid (\tikztotarget);
        \grd@start
        \pgfmathsetmacro{\grd@xa}{\the\pgf@x/1cm}
        \pgfmathsetmacro{\grd@ya}{\the\pgf@y/1cm}
        \grd@target
        \pgfmathsetmacro{\grd@xb}{\the\pgf@x/1cm}
        \pgfmathsetmacro{\grd@yb}{\the\pgf@y/1cm}
        \pgfmathsetmacro{\grd@xc}{\grd@xa + \pgfkeysvalueof{/tikz/grid with coordinates/major step}}
        \pgfmathsetmacro{\grd@yc}{\grd@ya + \pgfkeysvalueof{/tikz/grid with coordinates/major step}}
        \foreach \x in {\grd@xa,\grd@xc,...,\grd@xb}
        \node[anchor=north] at (\x,\grd@ya) {\pgfmathprintnumber{\x}};
        \foreach \y in {\grd@ya,\grd@yc,...,\grd@yb}
        \node[anchor=east] at (\grd@xa,\y) {\pgfmathprintnumber{\y}};
      }
    }
  },
  minor help lines/.style={
    help lines,
    step=\pgfkeysvalueof{/tikz/grid with coordinates/minor step}
  },
  major help lines/.style={
    help lines,
    line width=\pgfkeysvalueof{/tikz/grid with coordinates/major line width},
    step=\pgfkeysvalueof{/tikz/grid with coordinates/major step}
  },
  grid with coordinates/.cd,
  minor step/.initial=.2,
  major step/.initial=1,
  major line width/.initial=2pt,
}
\makeatother

%define new environment
% \newenvironment{definition}[1][定义]{\begin{trivlist}
% \item[\hskip \labelsep {\bfseries #1}]}{\end{trivlist}}
\newtheorem{definition}{定义}


\newif\ifUseComments
\UseCommentstrue %\UseCommentstrue-打印注释,\UseCommentsfalse-不打印注释
\definecolor{ao(english)}{rgb}{0.0, 0.5, 0.0}
\newcommand{\comments}[1]{
    \ifUseComments
    {\scriptsize
    \color{ao(english)}
    \begin{itemize}[label = \%, itemsep=-1ex, leftmargin=3ex,topsep = -2pt]
        #1
    \end{itemize}}
    \fi
}

\newcommand{\demonstrate}[1]{\textcolor{red}{#1}}
\newcommand{\futureworks}[1]{
    {\scriptsize\color{red}
        #1
    }
}

